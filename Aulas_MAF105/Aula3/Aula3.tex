\documentclass[14pt,aspectratio=1610]{beamer}

\usepackage[brazil]{babel}
\usepackage[utf8]{inputenc}
%\UseRawInputEncoding
\usepackage[T1]{fontenc}
\usepackage{Sweave}
\usepackage{animate}
\usepackage{amsbsy}
\usepackage{amsfonts}
\usepackage{amsmath}
\usepackage{amssymb}
\usepackage{amsthm}
\usepackage[toc,page,title,titletoc]{appendix}
\usepackage[fixlanguage]{babelbib}
%\usepackage[pdftex]{color}
\usepackage{dsfont}
\usepackage{esvect}
\usepackage[labelfont=bf]{caption}
\usepackage{float}
\usepackage[Glenn]{fncychap}%Sonny %Conny %Lenny %Glenn %Renje %Bjarne %Bjornstrup
%\usepackage{geometry, calc, color, setspace}%
%\geometry{a4paper, headsep=1.0cm, footskip=1cm, lmargin=3cm, rmargin=2cm, tmargin=3cm, bmargin=2cm}
\usepackage{graphicx}
\usepackage{indentfirst}%Para indentar os parágrafos automáticamente
\usepackage{lipsum}
\usepackage{longtable}
\usepackage{mathtools}
\usepackage{listings}%Inserir codigo do R no latex
\usepackage{slashbox}
\usepackage{multirow}
\usepackage{multicol}
\usepackage{natbib}
\bibliographystyle{abbrvnat3}
\usepackage[figuresright]{rotating}
\usepackage{spalign}
%\usepackage{pgfpages}
\usepackage{pgfplots}
\usepackage{tikz}
\usepackage{color, colortbl}
\usepackage{ragged2e}%para justificar o texto dentro de algum ambiente
\definecolor{Gray}{gray}{0.9}
\definecolor{LightCyan}{rgb}{0.88,1,1}


\usepackage[all]{xy}
\usepackage{hyperref,bookmark}
\hypersetup{
  colorlinks=true,
  linkcolor=blue,
  citecolor=red,
  filecolor=blue,
  urlcolor=blue,
}

\usetheme{Dresden}
%\usecolortheme[RGB={193,0,0}]{structure}

%\setbeamertemplate{footline}[frame number]
%\setbeamertemplate{footline}[text line]{%
%  \parbox{\linewidth}{\vspace*{-8pt}\hfill\date{}\hfill\insertshortauthor\hfill\insertpagenumber}}
\beamertemplatenavigationsymbolsempty
\renewcommand{\vec}[1]{\mbox{\boldmath$#1$}}
\newtheorem{Teorema}{Teorema}
\newtheorem{Proposicao}{Proposição}
\newtheorem{Definicao}{Definição}
\newtheorem{Corolario}{Corolário}
\newtheorem{Demonstracao}{Demonstração}
\newcommand{\bx}{\ensuremath{\bar{x}}}
\newcommand{\Ho}{\ensuremath{H_{0}}}
\newcommand{\Hi}{\ensuremath{H_{1}}}

\apptocmd{\frame}{}{\justifying}{} % Allow optional arguments after frame.

\title{MAF 105 - Iniciação à Estatística}
\author{Prof. Fernando de Souza Bastos}
\institute{Instituto de Ciências Exatas e Tecnológicas\texorpdfstring{\\ Universidade Federal de Viçosa}{}\texorpdfstring{\\ Campus UFV - Florestal}{}}
\date[\today]{}
\newcommand\mytext{Análise Bidimensional}
\newcommand\mytextt{Fernando de Souza Bastos}
\makeatletter
\setbeamertemplate{footline}
{
  \leavevmode%
  \hbox{%
  \begin{beamercolorbox}[wd=.333333\paperwidth,ht=2.25ex,dp=1ex,center]{author in head/foot}%
    \usebeamerfont{author in head/foot}\mytext
  \end{beamercolorbox}%
  \begin{beamercolorbox}[wd=.333333\paperwidth,ht=2.25ex,dp=1ex,center]{title in head/foot}%
    \usebeamerfont{title in head/foot}\mytextt
  \end{beamercolorbox}%
  % \begin{beamercolorbox}[wd=.333333\paperwidth,ht=2.25ex,dp=1ex,right]{date in head/foot}%
  %   \usebeamerfont{date in head/foot}\insertshortdate{}\hspace*{2em}
  %   \insertframenumber{} / \inserttotalframenumber\hspace*{2ex} 
  % \end{beamercolorbox}
  }%
  \vskip0pt%
}
\makeatother


\providecommand{\arcsin}{} \renewcommand{\arcsin}{\hspace{2pt}\textrm{arcsen}}
\providecommand{\sin}{} \renewcommand{\sin}{\hspace{2pt}\textrm{sen}}
%\newtheorem{Teorema}{Teorema}
%\newtheorem{Proposicao}{Proposição}
%\newtheorem{Definicao}{Definição}
%\newtheorem{Corolario}{Corolário}
%\newtheorem{Demonstracao}{Demonstração}

% Layout da pagina
\hypersetup{pdfpagelayout=SinglePage}
\begin{document}
\Sconcordance{concordance:Aula3.tex:Aula3.Rnw:%
1 584 1 1 15 1 4 34 1 1 15 1 4 18 1 1 13 1 3 5 1 1 13 1 3 5 1 1 13 1 3 %
5 1 1 13 1 3 94 1}


\frame{\titlepage}

\begin{frame}{}
\frametitle{\bf Sumário}
\tableofcontents
\end{frame}

\section{Análise Bidimensional}
\begin{frame}{}
\frametitle{Introdução}
\begin{block}{}
\justifying
Até agora vimos como organizar e resumir informações pertinentes a uma única
variável (ou a um conjunto de dados), mas freqüentemente estamos interessados em
analisar o comportamento conjunto de duas ou mais variáveis aleatórias.
\end{block}
\end{frame}

\begin{frame}{}
\frametitle{Introdução}
\begin{block}{}
\justifying
Os dados aparecem na forma de uma matriz, usualmente com as colunas indicando as variáveis e as linhas os indivíduos (ou elementos).
\end{block}
\end{frame}

\begin{frame}{}
\frametitle{Introdução}
\begin{block}{}
\justifying
\begin{table}[htp]
\begin{tabular}{c|c|c|c|c|c|c}
\hline
\multirow{2}{*}{Indivíduo}&\multicolumn{6}{c}{Variável}\\
\cline{2-7}
      &$X_{1}$&$X_{2}$&$\cdots$&$X_{j}$&$\cdots$&$X_{p}$\\
\hline
1       &$X_{11}$&$X_{12}$&$\cdots$&$X_{1j}$&$\cdots$&$X_{1p}$\\
2       &$X_{21}$&$X_{22}$&$\cdots$&$X_{2j}$&$\cdots$&$X_{2p}$\\
$\vdots$&$\vdots$&$\vdots$&$\vdots$&$\vdots$&$\vdots$&$\vdots$\\
i       &$X_{i1}$&$X_{i2}$&$\cdots$&$X_{ij}$&$\cdots$&$X_{ip}$\\
$\vdots$&$\vdots$&$\vdots$&$\vdots$&$\vdots$&$\vdots$&$\vdots$\\
n       &$X_{n1}$&$X_{n2}$&$\cdots$&$X_{nj}$&$\cdots$&$X_{np}$\\
\hline
\end{tabular}
\end{table}
\end{block}
\end{frame}

\begin{frame}{}
\frametitle{}
% latex table generated in R 3.4.0 by xtable 1.8-2 package
% Mon Jun 05 18:03:42 2017
\newcolumntype{g}{>{\columncolor{Gray}}c}
\begin{table}[ht]
\centering
\caption{Tabela de dados MEPS 2001} 
\resizebox*{0.8\textwidth}{!}{%
\begin{tabular}{cgcgcgcgcgcgcgcgcgcgcgcg}
  \hline
 Obs.& Damb&  Ind & lDamb & Idade & Sexo & Educ & Blhisp & Totcr & Ins & Renda  \\ 
\hline
  50 & 0    &  0   & 0.000 & 4.5   & 1    & 12   & 0      & 2     & 0   & 24.480  \\ 
  51 & 996  &  1   & 6.904 & 6.4   & 1    & 9    & 0      & 2     & 0   & 30.000  \\ 
  52 & 2765 &  1   & 7.925 & 3.7   & 0    & 16   & 0      & 1     & 0   & 47.000  \\ 
  53 & 89   &  1   & 4.489 & 6.3   & 0    & 14   & 0      & 0     & 0   & 50.013  \\ 
  54 & 568  &  1   & 6.342 & 5.7   & 1    & 16   & 0      & 0     & 0   & 50.013  \\ 
  55 & 0    &  0   & 0.000 & 3.9   & 0    & 16   & 0      & 0     & 1   & 14.424  \\ 
  56 & 204  &  1   & 5.318 & 4.0   & 1    & 16   & 0      & 1     & 1   & 28.849  \\ 
  57 & 0    &  0   & 0.000 & 2.7   & 0    & 12   & 0      & 0     & 0   & 39.000  \\ 
  58 & 108  &  1   & 4.682 & 2.5   & 1    & 12   & 0      & 0     & 0   & 18.700  \\ 
  59 & 1293 &  1   & 7.165 & 3.0   & 1    & 12   & 0      & 0     & 0   & 5.667   \\ 
  60 & 133  &  1   & 4.890 & 5.0   & 1    & 14   & 0      & 0     & 1   & 1.370   \\ 
   \hline
\end{tabular}
}
\end{table}
\end{frame}

\begin{frame}{}
\frametitle{}
\begin{block}{}
\justifying
Quando consideramos duas variáveis (ou dois conjuntos de dados), podemos ter
três situações:
\begin{enumerate}
\item as duas variáveis são qualitativas;
\item as duas variáveis são quantitativas; e
\item uma variável é qualitativa e outra é quantitativa.
\end{enumerate}
As técnicas de análise de dados nas três situações são diferentes.

\end{block}
\end{frame}

\section{Variáveis Qualitativas}
\begin{frame}{}
\frametitle{}
\begin{block}{}
\justifying
Suponha que queiramos analisar o comportamento conjunto das variáveis
$X:$ grau de instrução e $Y:$ região de procedência, cujas observações estão contidas
na Tabela abaixo:
\begin{table}[htp]
\caption{Distribuição Conjunta das frequências das variáveis $Y$ e $X.$}
\begin{tabular}{c|c|c|c|c|c}
\hline
\multirow{2}{*}{\backslashbox{Y}{X}}&Ensino     &Ensino&Ensino&\multirow{2}{*}{Total}\\
                                    &Fundamental&Médio &Superior&\\
                                    \hline
Capital &4&5&2&11\\
Interior&3&7&2&12\\
Outra   &5&6&2&13\\
\hline
Total   &12&18&6&36\\
\hline
\end{tabular}
\end{table}
\end{block}
\end{frame}

\begin{frame}{}
\frametitle{}
\begin{block}{}
\justifying
\begin{itemize}
\item 4 indivíduos procedem da capital e possuem o ensino fundamental;\pause
\item Na última coluna, está representada a frequência absoluta da
variável $Y;$\pause
\item Na última linha está representada a frequência absoluta da variável $X;$\pause
\item As frequências absolutas (parte interna da tabela) são chamadas de
frequências absolutas conjuntas entre $X$ e $Y.$
\end{itemize}
\end{block}
\end{frame}

\begin{frame}{}
\frametitle{}
\begin{block}{}
\justifying
Em vez de trabalharmos com as freqüências absolutas, podemos construir tabelas
com as freqüências relativas (proporções), como foi feito no caso unidimensional.
Mas aqui existem três possibilidades de expressarmos a proporção de cada casela:
\begin{itemize}
\item em relação ao total geral;\pause
\item em relação ao total de cada linha;\pause
\item ou em relação ao total de cada coluna.
\end{itemize}
De acordo com o objetivo do problema em estudo, uma delas será a mais conveniente.

\end{block}
\end{frame}

\begin{frame}{}
\frametitle{}
\begin{block}{}
\justifying
\begin{table}[htp]
\caption{Distribuição conjunta das proporções (em porcentagem) em relação ao 
total geral das variáveis  $Y$ e $X$.}
\begin{tabular}{c|c|c|c|c|c}
\hline
\multirow{2}{*}{\backslashbox{Y}{X}}&Ensino     &Ensino&Ensino&\multirow{2}{*}{Total}\\
                                    &Fundamental&Médio &Superior&\\
                                    \hline
Capital &$11\%$&$14\%$&$6\%$&$31\%$\\
Interior&$ 8\%$&$19\%$&$6\%$&$33\%$\\
Outra   &$14\%$&$17\%$&$5\%$&$36\%$\\
\hline
Total   &$33\%$&$50\%$&$17\%$&$100\%$\\
\hline
\end{tabular}
\end{table}
\end{block}
\end{frame}

\begin{frame}{}
\frametitle{}
\begin{block}{}
\justifying
Podemos, então, afirmar que $11\%$ dos empregados vêm da capital e têm o ensino fundamental. Os totais nas margens fornecem as distribuições unidimensionais de cada 
uma das variáveis. Por exemplo, $31\%$ dos indivíduos vêm da capital, $33\%$ do interior e $36\%$ de outras regiões.
\end{block}
\end{frame}

\begin{frame}{}
\frametitle{}
\begin{block}{}
\justifying
\begin{table}[htp]
\caption{Distribuição conjunta das proporções (em porcentagem) em relação aos 
totais de cada coluna.}
\begin{tabular}{c|c|c|c|c|c}
\hline
\multirow{2}{*}{\backslashbox{Y}{X}}&Ensino     &Ensino&Ensino&\multirow{2}{*}{Total}\\
                                    &Fundamental&Médio &Superior&\\
                                    \hline
Capital &$33\%$&$28\%$&$33\%$&$31\%$\\
Interior&$25\%$&$39\%$&$33\%$&$33\%$\\
Outra   &$42\%$&$33\%$&$34\%$&$36\%$\\
\hline
Total   &$100\%$&$100\%$&$100\%$&$100\%$\\
\hline
\end{tabular}
\end{table}
\end{block}
\end{frame}

\begin{frame}{}
\frametitle{}
\begin{block}{}
\justifying
Entre os empregados com ensino médio:
\begin{itemize}
\item $28\%$ vêm da capital;\pause
\item $39\%$ vêm do interior;\pause
\item $33\%$ vêm de outros locais.
\end{itemize}
Esse tipo de tabela serve para comparar a distribuição da procedência dos
indivíduos conforme o grau de instrução.
\end{block}
\end{frame}

\begin{frame}{}
\frametitle{}
\begin{block}{}
\justifying
\begin{table}[htp]
\caption{Distribuição conjunta das proporções (em porcentagem) em relação aos 
totais de cada linha.}
\begin{tabular}{c|c|c|c|c|c}
\hline
\multirow{2}{*}{\backslashbox{Y}{X}}&Ensino     &Ensino&Ensino&\multirow{2}{*}{Total}\\
                                    &Fundamental&Médio &Superior&\\
                                    \hline
Capital &$36.4\%$&$45.4\%$&$18.2\%$&$100\%$\\
Interior&$ 25 \%$&$58.3\%$&$16.7\%$&$100\%$\\
Outra   &$38.5\%$&$46.1\%$&$15.4\%$&$100\%$\\
\hline
Total   &$33\%$&$50\%$&$17\%$&$100\%$\\
\hline
\end{tabular}
\end{table}
\end{block}
\end{frame}

\begin{frame}{}
\frametitle{}
\begin{block}{}
\justifying
Entre os empregados do interior:
\begin{itemize}
\item $25\%$ têm Ensino Fundamental;\pause
\item $58.3\%$ têm Ensino médio;\pause
\item $16.7\%$ têm ensino superior.
\end{itemize}
Esse tipo de tabela serve para comparar a distribuição do grau de instrução dos
indivíduos conforme a procedência.
\end{block}
\end{frame}

\section{Associação entre Variáveis Qualitativas}
\begin{frame}{}
\frametitle{}
\begin{block}{}
\justifying
Um dos principais objetivos de se construir uma distribuição conjunta de duas
variáveis qualitativas é descrever a associação entre elas, isto é, queremos conhecer o
grau de dependência entre elas, de modo que possamos prever melhor o resultado de
uma delas quando conhecermos a realização da outra.
\end{block}
\end{frame}

\begin{frame}{}
\frametitle{}
\begin{block}{}
\justifying
Por exemplo, se quisermos estimar qual a renda média de uma família moradora
da cidade de São Paulo, a informação adicional sobre a classe social a que ela pertence
nos permite estimar com maior precisão essa renda, pois sabemos que existe uma
dependência entre as duas variáveis: renda familiar e classe social.
\end{block}
\end{frame}

\begin{frame}{}
\frametitle{}
\begin{block}{}
\justifying
\begin{itemize}
\item Suponhamos que uma pessoa seja sorteada ao acaso e devamos adivinhar o sexo dessa pessoa. Existe sexo mais provável?\pause
\item E se a mesma pergunta fosse feita, porém fosse dito que a pessoa sorteada trabalha na indústria siderúrgica, qual a resposta mais provável?\pause
\end{itemize}
Ou seja, há um grau de dependência grande entre as variáveis sexo e ramo de atividade.
\end{block}
\end{frame}

\begin{frame}{}
\frametitle{}
\begin{block}{}
\justifying
Queremos verificar se existe ou não associação entre o sexo e a carreira escolhida por 200 alunos de Economia e Administração.
\begin{table}[htb]
\caption{Distribuição conjunta de alunos segundo o sexo $(X)$ e o curso escolhido 
$(Y).$}
\begin{tabular}{c|c|c|c}
\backslashbox{Y}{X}&Masculino&Feminino&Total\\
\hline
Economia     &85&35&120\\
Administração&55&25&80\\
\hline
Total&140&60&200\\
\hline
\end{tabular}

{\raggedright \footnotesize{Fonte: \cite{Morettin09}.}}\pause
\end{table}
Inicialmente, verificamos que fica muito difícil tirar alguma conclusão, devido à diferença entre os totais marginais.
\end{block}
\end{frame}

\begin{frame}{}
\frametitle{}
\vspace{-0.5cm}
\begin{block}{}
\justifying
Devemos, pois, construir as proporções segundo as linhas ou as colunas para podermos fazer comparações. \vspace{-0.7cm}
\begin{table}[htb]
\caption{Distribuição conjunta das proporções de alunos segundo o sexo $(X)$ e o curso escolhido 
$(Y).$}
\begin{tabular}{c|c|c|c}
\backslashbox{Y}{X}&Masculino&Feminino&Total\\
\hline
Economia     &$61\%$&$58\%$&$60\%$\\
Administração&$39\%$&$42\%$&$40\%$\\
\hline
Total&$100\%$&$100\%$&$100\%$\\
\hline
\end{tabular}
\pause
\end{table}
\vspace{-0.5cm}
observar que, independentemente do sexo, $60\%$ preferem Economia e $40\%$ preferem Administração. Não havendo dependência entre as variáveis, esperaríamos essas mesmas proporções para cada sexo.
\end{block}
\end{frame}

\begin{frame}{}
\frametitle{Outro exemplo:}
\begin{block}{}
\justifying
\begin{table}[htb]
\caption{Distribuição conjunta das proporções de alunos segundo o sexo $(X)$ e o curso escolhido 
$(Y).$}
\begin{tabular}{c|c|c|c}
\backslashbox{Y}{X}&Masculino&Feminino&Total\\
\hline
Física          &$100(71\%)$&$20(33\%)$&$120(60\%)$\\
Ciências Sociais&$40(29\%)$&$40(67\%)$&$80(40\%)$\\
\hline
Total&$140(100\%)$&$60(100\%)$&$200(100\%)$\\
\hline
\end{tabular}

{\raggedright \footnotesize{Fonte: \cite{Morettin09}.}}
\end{table}
\end{block}
\end{frame}

\begin{frame}{}
\frametitle{}
\begin{block}{}
Verifique se a criação de determinado tipo de cooperativa está associada com algum fator regional:
\begin{table}[h!]
\parbox{10cm}{\caption{Cooperativas autorizadas a funcionar por tipo e estado, junho de 1974.}}
\resizebox*{0.8\textwidth}{!}{%
\begin{tabular}{c|c|c|c|c|c}
\hline
\multirow{2}{*}{Estado}&\multicolumn{4}{c|}{Tipos de Cooperativa}       &\multirow{2}{*}{Total}\\
\cline{2-5}
                                      &Consumidor  &Produtor   &    Escola&Outras  &                      \\
\hline                       
São Paulo              &$214(33\%)$&$237(37\%)$&$78(12\%) $&$119(18\%)$&$648(100\%)$\\
Paraná                 &$51 (17\%)$&$102(34\%)$&$126(42\%)$&$22 ( 7\%)$&$301(100\%)$\\
Rio G. do Sul          &$111(18\%)$&$304(51\%)$&$139(23\%) $&$48(8\%)$&$602(100\%)$\\
\hline
Total          &$376(24\%)$&$643(42\%)$&$343(22\%) $&$189(12\%)$&$1.551(100\%)$\\
\hline
\end{tabular}
}
\end{table}
\end{block}
\end{frame}

\begin{frame}{}
\frametitle{}
\begin{block}{}
\justifying
\begin{table}[h!]
\parbox{10cm}{\caption{Valores esperados ao assumir independência entre as variáveis.}}
\resizebox*{0.8\textwidth}{!}{%
\begin{tabular}{c|c|c|c|c|c}
\hline
\multirow{2}{*}{Estado}&\multicolumn{4}{c|}{Tipos de Cooperativa}       &\multirow{2}{*}{Total}\\
\cline{2-5}
                                      &Consumidor  &Produtor   &    Escola&Outras  &                      \\
\hline                       
São Paulo              &$157(24\%)$&$269(42\%)$&$143(22\%) $&$79(12\%)$&$648(100\%)$\\
Paraná                 &$73 (24\%)$&$124(42\%)$&$67(22\%)$&$37 ( 12\%)$&$301(100\%)$\\
Rio G. do Sul          &$146(24\%)$&$250(42\%)$&$133(22\%) $&$73(12\%)$&$602(100\%)$\\
\hline
Total          &$376(24\%)$&$643(42\%)$&$343(22\%) $&$189(12\%)$&$1.551(100\%)$\\
\hline
\end{tabular}
}
\end{table}
\end{block}
\end{frame}

\section{Associação entre Variáveis Quantitativas}
\begin{frame}[fragile]{}
\frametitle{}
\vspace{-0.5cm}
\begin{tabular}{cl}  
        \begin{tabular}{c}
          \parbox{0.5\linewidth}{
          \begin{table}[h!]
  \caption{Anos de Serviço $(X)$ versus\\ Nº de Clientes $(Y)$}
  \begin{tabular}{ccc}
  \hline
  Agente&X&Y\\
  \hline
  A&2&48\\
  B&4&56\\
  C&5&64\\
  D&6&60\\
  E&6&65\\
  F&6&63\\
  G&7&67\\
  H&8&70\\
  I&8&71\\
  J&10&72\\
  \hline
  \hline
  \end{tabular}
  \end{table}
          }
        \end{tabular}
           & 
        \begin{tabular}{l}
          \setkeys{Gin}{width=0.4\linewidth}
\includegraphics{Aula3-001}
         \end{tabular}  \\
\end{tabular}
\end{frame}

\begin{frame}[fragile]{}
\frametitle{}
\vspace{-0.5cm}
\begin{tabular}{cl}  
        \begin{tabular}{c}
          \parbox{0.5\linewidth}{
          \begin{table}[h!]
  \caption{Renda bruta mensal $(X)$ e porcentagem da renda gasta em saúde $(Y).$}
  \begin{tabular}{ccc}
  \hline
  Família&X&Y\\
  \hline
  A&12&7.2\\
  B&16&7.4\\
  C&18&7.0\\
  D&20&6.5\\
  E&28&6.6\\
  F&30&6.7\\
  G&40&6.0\\
  H&48&5.6\\
  I&50&6.0\\
  J&54&5.5\\
  \hline
  \hline
  \end{tabular}
  \end{table}
          }
        \end{tabular}
           & 
        \begin{tabular}{l}
          \setkeys{Gin}{width=0.4\linewidth}
\includegraphics{Aula3-002}
         \end{tabular}  \\
\end{tabular}
\end{frame}

\begin{frame}{}
\frametitle{}
\begin{block}{}
\begin{figure}[H]
    \centering
    \includegraphics[scale=0.5]{Figuras/correlacao}
    \caption{\cite{Morettin09}}
    %\label{figRotulo}
  \end{figure}
\end{block}
\end{frame}

\begin{frame}[fragile]{}
\begin{center}
\setkeys{Gin}{width=0.6\linewidth}
\includegraphics{Aula3-003}
\end{center}
\end{frame}

\begin{frame}[fragile]{}
\begin{center}
\setkeys{Gin}{width=0.6\linewidth}
\includegraphics{Aula3-004}
\end{center}
\end{frame}

\begin{frame}[fragile]{}
\begin{center}
\setkeys{Gin}{width=0.6\linewidth}
\includegraphics{Aula3-005}
\end{center}
\end{frame}

\begin{frame}[fragile]{}
\begin{center}
\setkeys{Gin}{width=0.6\linewidth}
\includegraphics{Aula3-006}
\end{center}
\end{frame}

\begin{frame}{}
\frametitle{}
\vspace{-0.5cm}
% Table generated by Excel2LaTeX from sheet 'Plan1'
\begin{table}[htbp]
  \centering
  \caption{Cálculo do coeficiente de correlação.}
  \resizebox*{0.8\textwidth}{!}{%
    \begin{tabular}{c|c|c|c|c|c|c|c}
    \hline
    Agente & Anos  & Clientes & $x-\bar{x}$ & $y-\bar{y}$ & $\dfrac{x-\bar{x}}{dp(x)}=z_{x}$ & $\dfrac{y-\bar{y}}{dp(y)}=z_{y}$ & $z_{x}.z_{y}$ \\
    \hline
    A     & 2     & 48    & -3.7  & -8.5  & -1.54 & -1.05 & 1.617 \\
    B     & 3     & 50    & -2.7  & -6.5  & -1.12 & -0.8  & 0.896 \\
    C     & 4     & 56    & -1.7  & -0.5  & -0.71 & -0.06 & 0.043 \\
    D     & 5     & 52    & -0.7  & -4.5  & -0.29 & -0.55 & 0.160 \\
    E     & 4     & 43    & -1.7  & -13.5 & -0.71 & -1.66 & 1.179 \\
    F     & 6     & 60    & 0.3   & 3.5   & 0.12  & 0.43  & 0.052 \\
    G     & 7     & 62    & 1.3   & 5.5   & 0.54  & 0.68  & 0.367 \\
    H     & 8     & 58    & 2.3   & 1.5   & 0.95  & 0.18  & 0.171 \\
    I     & 8     & 64    & 2.3   & 7.5   & 0.95  & 0.92  & 0.874 \\
    J     & 10    & 72    & 4.3   & 15.5  & 1.78  & 1.91  & 3.400 \\
    \hline
    Total & 57    & 565   & 0     & 0     &       &       & 8.759 \\
    \hline
    \end{tabular}%
  \label{tab:addlabel}%
}
\end{table}%
$$Cor={\displaystyle \dfrac{1}{n}\sum_{i=1}^{n}\Biggl(\dfrac{x_{i}-\bar{x}}{dp(X)}\Biggl)\Biggl(\dfrac{y_{i}-\bar{y}}{dp(Y)}\Biggl)}=\dfrac{8.759}{10}=0.8759$$
\end{frame}

\begin{frame}{}
\frametitle{}
\vspace{-0.6cm}
\begin{block}{}
\justifying
Não é difícil provar que o coeficiente de correlação satisfaz:
$$-1\leq cor(X,Y)\leq 1$$
\end{block}
\pause
\vspace{-0.8cm}
\begin{block}{}
\justifying
{\bf DEF:} Dados $n$ pares de valores $(x_{1}, y_{1}), \cdots, (x_{n}, y_{n}),$ chamaremos de covariância entre as duas variáveis $X$ e $Y$ a igualdade:
$$cov(X,Y)={\displaystyle \sum_{i=1}^{n}\dfrac{(x_{i}-\bar{x})(y_{i}-\bar{y})}{n}}$$
\end{block}
\pause
\vspace{-0.8cm}
\begin{block}{}
\justifying
Com a definição acima, o coeficiente de correlação pode ser escrito como:
$$cor(X,Y)=\dfrac{cov(X,Y)}{dp(X)dp(Y)}$$
\end{block}
\end{frame}

\begin{frame}{}
\frametitle{}
A covariância mede a relação linear entre duas variáveis. A covariância é semelhante à correlação entre duas variáveis, no entanto, elas diferem nas seguintes maneiras:
\begin{itemize}
\item Os coeficientes de correlação são padronizados. Assim, um relacionamento linear perfeito resulta em um coeficiente de correlação 1. A correlação mede tanto a força como a direção da relação linear entre duas variáveis.\pause
\item Os valores de covariância não são padronizados. Como os dados não são padronizadas, é difícil determinar a força da relação entre as variáveis.
\end{itemize}
\end{frame}

\section{Associação entre Variáveis Qualitativas e Quantitativas}
\begin{frame}{}
\frametitle{}
Associação entre Variáveis Qualitativas e Quantitativas
\begin{figure}[H]
    \centering
    \includegraphics[scale=0.5]{Figuras/boxplot}
    \caption{Box plots de salário segundo grau de instrução. \cite{Morettin09}}
    %\label{figRotulo}
  \end{figure}
\end{frame}

\begin{frame}{}
\frametitle{}
\begin{figure}[H]
    \centering
    \includegraphics[scale=0.5]{Figuras/boxplot2}
    \caption{Box plots de salário segundo região de procedência. \cite{Morettin09}}
    %\label{figRotulo}
  \end{figure}
\end{frame}

\begin{frame}{}
\frametitle{Referências Bibliográficas}
\bibliography{bibliografia}
\end{frame}

\end{document}
