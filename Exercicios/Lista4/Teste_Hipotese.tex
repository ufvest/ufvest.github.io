\documentclass{report}
\usepackage[utf8]{inputenc}
\usepackage[T1]{fontenc}
\usepackage[brazil]{babel}
\usepackage{graphicx}
\usepackage{amsfonts}
\usepackage{amssymb}
\usepackage{amsmath}
\usepackage{multicol}
\usepackage{ifthen}
\newboolean{firstanswerofthechapter}  
\usepackage{xcolor}
\colorlet{lightcyan}{cyan!40!white}
\usepackage{chngcntr}
\usepackage{stackengine}
\usepackage{tasks}
\usepackage{multirow}
\usepackage{float}
\newlength{\longestlabel}
\settowidth{\longestlabel}{\bfseries viii.}
%\settasks{counter-format={tsk[r].}, label-format={\bfseries}, label-width=\longestlabel,
    %item-indent=0pt, label-offset=2pt, column-sep={10pt}}
		
\setcounter{secnumdepth}{0} \setlength{\topmargin}{0cm}
\setlength{\headsep}{-0.3cm} \setlength{\textwidth}{17.5cm}
\setlength{\textheight}{23cm} \setlength{\oddsidemargin}{-0.8cm}
\setlength{\evensidemargin}{0cm} \setlength{\footskip}{-1.5cm}

		
\usepackage[lastexercise,answerdelayed]{exercise}
%\counterwithin{Exercise}{chapter}
%\counterwithin{Answer}{chapter}
%\renewcounter{Exercise}[chapter]
%\newcommand{\QuestionNB}{\bfseries\arabic{Question}.\ }
%\renewcommand{\ExerciseName}{Exercício}
%\renewcommand{\ExerciseHeader}{\noindent\def\stackalignment{l}% code from https://tex.stackexchange.com/a/195118/101651
    %\stackunder[0pt]{\colorbox{cyan}{\textcolor{white}{\textbf{\LARGE\ExerciseHeaderNB\;\large\ExerciseName}}}}{\textcolor{lightcyan}{\rule{\linewidth}{2pt}}}\medskip}
\renewcommand{\ExerciseName}{Exercícios}
\renewcommand{\ExerciseHeader}{\noindent\def\stackalignment{l}% code from https://tex.stackexchange.com/a/195118/101651
    \stackunder[0pt]{\colorbox{cyan}{\textcolor{white}{\textbf{\large\ExerciseName}}}}{\textcolor{lightcyan}{\rule{\linewidth}{2pt}}}\medskip}
%\renewcommand{\AnswerName}{Exercises}
%\renewcommand{\AnswerHeader}{\ifthenelse{\boolean{firstanswerofthechapter}}%
    %{\bigskip\noindent\textcolor{cyan}{\textbf{CHAPTER \thechapter}}\newline\newline%
        %\noindent\bfseries\emph{\textcolor{cyan}{\AnswerName\ \ExerciseHeaderNB, page %
                %\pageref{\AnswerRef}}}\smallskip}
    %{\noindent\bfseries\emph{\textcolor{cyan}{\AnswerName\ \ExerciseHeaderNB, page \pageref{\AnswerRef}}}\smallskip}}
%\setlength{\QuestionIndent}{16pt}

\begin{document}

\vspace*{-2cm}

\begin{center}
\begin{minipage}[s]{2cm}
\hspace{-1.3cm}\includegraphics[scale=1.0]{Figuras/brasaoufv.eps}
\end{minipage}
\begin{minipage}[s]{13cm}
{\begin{center} {\sc \Large Universidade Federal de Vi\c{c}osa}\\
{\sc \large Instituto de Ci\^encias Exatas e Tecnológicas}\\
{\sc \large Campus UFV - Florestal}\\
\end{center}}
\end{minipage}\begin{minipage}[s]{2 cm}
%\includegraphics[width=2 cm]{logoimecc.eps}
\end{minipage}
\end{center}

\vspace{-0.3cm}

%\hline \hline \noindent

%%%%%%%%%%%%%%%%%%%%%%%%%%%%%%%%%%%%%%%%%%%%%%%%%%%%%%%%%%%%%%%%%%%%%%%%%%%

\medskip

\begin{center}

\underline{\underline{{\large{\sc Lista de Iniciação à Estatística - Testes de Hipóteses}}}}

\bigskip

{\large {\bf Prof. Fernando Bastos}}
%\bigskip
%
%%{\sc Data: $19/06/2018$}
\end{center}

\begin{Exercise}

\Question Trace uma curva normal e sombreie a área desejada obtendo então a informação.
\begin{tasks}
\task Área à direita de $Z = 1$
\task Área à esquerda de $Z = 1$
\task Área entre $Z = 0$ e $Z = 1,5$
\task Área entre $Z = -0,56$ e $Z = -0,2$
\task Área entre $Z = 0,5$ e $Z = 0,5$
\task Área entre $Z = 0$ e $Z = -2,5$
\end{tasks} 

\Question Usando a tabela da distribuição normal, determine os valores de $Z$ que correspondem às seguintes áreas:
\begin{tasks}
\task Área de 0,0505 à esquerda de Z.
\task Área de 0,0228 à direita de Z
\task Área de 0,0228 à esquerda de Z
\task 0,4772 entre 0 e $z.$
\end{tasks} 

\Question Consultando a tabela, determine a probabilidade de certo valor padronizado de Z estar entre $Z_{0} = -1,20$ e $Z_{1} = 2,00$. Desenhe o gráfico.

\Question Dado uma variável $X$ com distribuição normal de média 25 e desvio-padrão 2, determine os valores de Z para os seguintes valores $(x):$
\begin{tasks}
\task 23
\task 23,5
\task 24
\task 25,2
\task 25,5
\end{tasks}

\Question Determine a probabilidade de certo valor padronizado de Z estar entre $Z_{0} = -1,30$ e $Z_{1}=1.5$. Desenhe o gráfico.

\Question Uma população normal tem média 40 e desvio-padrão 3. Determine os valores da população correspondentes aos seguintes de Z:
\begin{tasks}
\task 	0,10
\task 	2,00
\task 	0,75
\task 	-3,00
\task 	-2,53
\end{tasks}

\newpage

\Question Explique com suas palavras, exemplificando, o significado de:
\begin{tasks}
\task teste de hipótese;
\task Hipótese nula e alternativa;
\task erros do tipo I e II; 
\task nível de significância.
\end{tasks}

\Question Enuncie a hipótese nula e a hipótese alternativa em cada um dos casos a seguir.
\begin{tasks}
\task A produção média de certo cereal é de 40 toneladas por hectare. Acredita-se que um novo tipo de adubo aumenta a produção média por hectare.
\task Um sindicato de empregados de certa categoria deseja verificar se a taxa de desemprego em certo município é maior que a taxa de $12\%$ observada seis meses antes.
\end{tasks}

\Question O fabricante de certa marca de suco informa que as embalagens de seu produto têm em média 500 ml, com desvio padrão igual a 10 ml. Tendo sido encontradas 
no mercado algumas embalagens com menos de 500 ml, suspeita-se que a informação do fabricante seja falsa. Para verificar se isto ocorre, um fiscal analisa uma amostra 
de 200 embalagens escolhidas aleatoriamente no mercado e constata que as mesmas contêm em média 498 ml. Considerando-se um nível de significância de $5\%,$ 
pode-se afirmar que o fabricante está mentindo? Calcule o valor da prova para esta amostra.

\Question A duração das lâmpadas produzidas por certo fabricante tem distribuição normal com média igual a 1200 horas e desvio padrão igual a 300 horas. O fabricante 
introduz um novo processo na produção das lâmpadas. Para verificar se o novo processo produz lâmpadas de maior duração, o fabricante observa 100 lâmpadas produzidas 
pelo novo processo e constata que as mesmas duram em média 1265 horas. Admitindo-se um nível de significância de $5\%,$ pode-se concluir que o novo processo produz 
lâmpadas com maior duração?

\Question O custo de produção de certo artigo numa localidade tem distribuição normal com média igual a $R\$42,00.$ Desenvolve-se uma política de redução de custos 
na empresa para melhorar a competitividade do referido produto no mercado. Observando-se os custos de 10 unidades deste produto, obtiveram-se os seguintes valores: 
34, 41, 36, 41, 29, 32, 38, 35, 33 e 30. Admitindo-se um nível de significância de $5\%,$ pode-se afirmar que o custo do produto considerado diminuiu? 

\Question O controle de qualidade das peças produzidas por certa fábrica exige que o diâmetro médio das mesmas seja 57 mm. Para verificar se o processo de produção 
está sob controle, observam-se os diâmetros de 10 peças, constatando-se os seguintes valores em mm: 56,5; 56,6; 57,3; 56,9; 57,1; 56,7; 57,1; 56,8; 57,1; 57,0. 
Admitindo-se um nível de significância de $5\%,$ pode-se concluir que o processo de produção está sob controle?

\Question Numa localidade, $32\%$ dos consumidores consomem determinado produto. Foi lançado no mercado da localidade um produto concorrente. Uma pesquisa 
realizada com 500 consumidores escolhidos ao acaso revelou que 145 dentre estes consomem o antigo produto. Pode-se concluir, num nível de significância de $2\%,$ que 
a preferência pelo produto antigo diminuiu com a entrada do concorrente no mercado? Calcule o valor da prova para esta amostra. 

\Question Sabe-se que $6\%$ das unidades de certo produto são substituídas gratuitamente por apresentar defeitos de fabricação. Para reduzir este percentual, o 
fabricante investiu na melhoria da qualidade do produto. Consta-se que 12 dentre 400 unidades vendidas tiveram que ser substituídas gratuitamente por apresentar defeitos 
de fabricação. Pode-se concluir, num nível de significância de $3\%,$ que a qualidade do produto melhorou?

\Question Suponha que o tempo necessário para que estudantes completem uma prova tenha distribuição normal com média 90 minutos e desvio padrão 15 minutos.
\begin{tasks}
\task Qual é a probabilidade do estudante terminar a prova em menos de 80 minutos?
\task Em mais de 120 minutos? 
\task Entre 75 e 85 minutos?
\task Qual é o tempo necessário para que $98\%$ dos estudantes terminem a prova?
\end{tasks}

\newpage

\Question Uma v.a. $X$ tem distribuição normal, com média $100$ e desvio padrão $10.$
\begin{tasks}
\task Qual a $P(90 < X < 110)$?
\task Se $\bar{X}$ for a média de uma amostra de 16 elementos retirados dessa população, calcule $P(90 < \bar{X} < 110)$.
\task Represente, num único gráfico, as distribuições de $X$ e $\bar{X}$.
\task Que tamanho deveria ter a amostra para que $P(90 < \bar{X}< 110) = 0,95$?
\end{tasks}

\Question Nas situações abaixo, escolha como hipótese nula, $H_0$, aquela que para você leva a um erro tipo I mais importante. Descreva quais os dois erros em 
cada caso.
\begin{tasks}
\task O trabalho de um operador de radar é detectar aeronaves inimigas. Quando surge alguma coisa estranha na tela, ele deve decidir entre as hipotéses:\\
1. está começando um ataque;\\
2. tudo bem, apenas uma leve interferência.
\task Num júri, um indivíduo está sendo julgado por um crime. As hipóteses sujeitas ao júri são:\\
1. o acusado é inocente;\\
2. o acusado é culpado.
\task Um pesquisador acredita que descobriu uma vacina contra resfriado. Ele irá conduzir uma pesquisa de laboratório para verificar a veracidade da afirmação. De acordo 
com o resultado, ele lançará ou não a vacina no mercado. As hipóteses que pode testar são:\\
1. a vacina é eficaz;\\
2. a vacina não é eficaz.
\end{tasks}

\Question Uma fábrica de automóveis anuncia que seus carros consomem, em média, 11 litros por 100 km, com desvio padrão de 0,8 litros. Uma revista resolve testar essa 
afirmação e analisa 35 automóveis dessa marca, obtendo 11,3 litros por 100 km como consumo médio (considerar distribução normal). O que a revista pode concluir 
sobre o anúncio da fábrica, no nível de 10\%?

\Question Duas máquinas, A e B, são usadas para empacotar pó de café. A experiência passada garante que o desvio padrão para ambas é de 10 g. Porém, suspeita-se 
que elas têm médias diferentes. Para verificar, sortearam-se duas amostras: uma com 25 pacotes da máquina A e outra com 16 pacotes da máquina B. As médias foram, 
respectivamente, $\bar x_A = 502,74g$ e $\bar x_B = 496,60 g$.  Com esses números, e com o nível de $5\%$, qual seria a coclusão do teste $H_0: \mu_A = \mu_B$?

\Question Uma fábrica de embalagens para produtos químicos está estudando dois processos para combater a corrosão de suas latas especiais. Para verificar o efeito dos 
tratamentos, foram usadas amostras cujos resultados estão no quadro abaixo (em porcentagem de corrosão eliminada). Qual seria a conclusão sobre os dois tratamentos?

\begin{tabular}{cccc}\\ \hline
Método & Amostra & Média & Desvio Padrão \\ \hline
A & 15 & 48 & 10 \\
B & 12 & 52 & 15 \\ \hline
\end{tabular}

\Question \item Para investigar a influência da opção profissional sobre o salário inicial de recém-formados, investigaram-se dois grupos de profissionais: um de liberais em geral 
e outro de formandos em Administração de Empresas. Com os resultados abaixo, expressos em salários mínimos, quais seriam suas conclusões?

\begin{tabular}{ccccccccc}\\ \hline
Liberais & 6,6 & 10,3 & 10,8 & 12,9 & 9,2 & 12,3 & 7,0 &  \\ \hline
Administradores & 8,1 & 9,8 & 8,7 & 10,0 & 10,2 & 8,2 & 8,7 & 10,1 \\ \hline
\end{tabular}

\newpage

\Question Os dados abaixo referem-se a medidas de determinada variável em 19 pessoas antes e depois de uma cirurgia. Verifique se as medidas pré e pós-operatórias 
apresentam a mesma média. Que suposições você faria para resolver o problema?

\begin{tabular}{c|c|c|c|c|c} \hline
 Pessoas & Pré & Pós & Pessoas & Pré & Pós \\ \hline
 1 & 50,0 & 42,0 & 10 & 40,0 & 50,0 \\
 2 & 50,0 & 42,0 & 11 & 50,0 & 48,0 \\
 3 & 50,0 & 78,0 & 12 & 75,0 & 52,0 \\
 4 & 87,5 & 33,0 & 13 & 92,5 & 74,0 \\
 5 & 32,5 & 96,0 & 14 & 38,0 & 47,5 \\
 6 & 35,0 & 82,0 & 15 & 46,5 & 49,0 \\
 7 & 40,0 & 44,0 & 16 & 50,0 & 58,0 \\
 8 & 45,0 & 31,0 & 17 & 30,0 & 42,0 \\
 9 & 62,5 & 87,0 & 18 & 35,0 & 60,0 \\
 10 &  &  & 19 & 39,4 & 28,0 \\ \hline
\end{tabular}

\Question Uma empresa deseja estudar o efeito de uma pausa de dez minutos para um cafezinho sobre a produtividade de seus trabalhadores. Para isso, sorteou seis operários, 
e contou o número de peças produzidas durante uma semana sem intervalo e uma semana com intervalo. Os resultados sugerem se há ou não melhora na produtividade? 
Caso haja melhora, qual deve ser o acréscimo médio de produção para todos os trabalhadores da fábrica?

\begin{tabular}{ccccccc}\hline
 Operário & 1 & 2 & 3 & 4 & 5 & 6 \\ \hline
 Sem intervalo & 23 & 35 & 29 & 33 & 43 & 32 \\
 Com intervalo & 28 & 38 & 29 & 37 & 42 & 30 \\ \hline
\end{tabular}

\Question Num levantamento feito com os operários da indústria mecânica, chegou-se aos seguintes números: salário médio = 3,64 salários mínimos e desvio padrão = 0,85 
salário mínimo. Suspeita-se que os salários de subclasse formada pelos torneiros mecânicos são diferentes dos salários do conjunto todo, tanto na média como na variância. 
Que conclusões você obteria se uma amostra de 25 torneiros apresentasse salário médio igual a 4,22 salários mínimos e desvio padrão igual a 1,25 salário mínimo?

\Question Um partido afirma que a porcentagem de votos masculinos a seu favor será de 10 \% a mais do que a porcentagem de votos femininos. Numa pesquisa feita entre 
400 homens, 170 votariam no partido, enquanto entre 625 mulheres, 194 lhe seriam favoráveis. A afirmação do partido é verdadeira ou não?

\Question De 400 moradores sorteados de uma grande cidade industrial, 300 são favoráveis a um projeto governamental, e de uma amostra de 160 moradores de uma cidade 
cuja principal atividade é o turismo, 120 são contra.
\begin{tasks}
\task Você diria que a diferença de opiniões nas duas cidades é estatisticamente significante?
\end{tasks}

\Question Para verificar o grau de adesão de uma nova cola para vidros, preparam-se dois tipos de montagem: cruzado (A), onde a cola é posta em forma de X, e quadrado (B), 
onde a cola é posta apenas nas quatro bordas. Os resultados da resistência para as duas amostras de 10 cada estão abaixo. Que tipo de conclusão poderia ser tirada?

\begin{tabular}{c|c|c|c|c|c|c|c|c|c|c} \hline
Método A & 16 & 14 & 19 & 18 & 19 & 20 & 15 & 18 & 17 & 18 \\ \hline
Método B & 13 & 19 & 14 & 17 & 21 & 24 & 10 & 14 & 13 & 15 \\ \hline
\end{tabular}

\Question Em um estudo para comparar os efeitos de duas dietas, A e B, sobre o crescimento, 6 ratos foram submetidos à dieta A, e 9 ratos à dieta B. Após 5 semanas, os ganhos em peso foram:

\begin{tabular}{c|c|c|c|c|c|c|c|c|c} \hline
Dieta A & 15 & 18 & 12 & 11 & 14 & 15 &  & &  \\ \hline
Dieta B & 11 & 11 & 12 & 16 & 12 & 13 & 8 & 10 & 13  \\ \hline
\end{tabular}
\begin{tasks}
\task Admitindo que temos duas amostras independentes de populações normais, teste a hipótese de que não há diferença entre as duas dietas, contra a alternativa que a 
dieta A é mais eficaz, usando o teste $t$ de Student, no nível de $\alpha = 0,01$.
\end{tasks}

\newpage

\Question Suponha que o tempo necessário para atendimento de clientes em uma central de atendimento telefônico siga uma distribuição normal de média de 8 minutos 
e desvio padrão de 2 minutos.
\begin{tasks}
\task Qual é a probabilidade de que um atendimento dure menos de 5 minutos?
\task E  mais do que 9,5 minutos?
\task E entre 7 e 10 minutos?
\task $75\%$ das chamadas telefônicas requerem pelo menos quanto tempo de atendimento?
\end{tasks}

\Question A distribuição dos pesos de coelhos criados numa granja pode muito bem ser representada por uma distribuição Normal, com média 5 kg e desvio padrão 0,9 kg. 
Um abatedouro comprará 5000 coelhos e pretende classificá-los de acordo com o peso do seguinte modo: $15\%$ dos mais leves como pequenos, os $50\%$ seguintes 
como médios, os $20\%$ seguintes como grandes e os $15\%$ mais pesados como extras. Quais os limites de peso para cada classificação?

\Question Uma enchedora automática de refrigerantes está regulada para que o volume médio de líquido em cada garrafa seja de $1000 cm^{3}$ e desvio padrão de $10 m^{3}$. 
Admita que o volume siga uma distribuição normal.
\begin{tasks}
\task Qual é a porcentagem de garrafas em que o volume de líquido é menor que $990 cm^{3}$?
\task Qual é a porcentagem de garrafas em que o volume de líquido não se desvia da média em mais do que dois desvios padrões?
\end{tasks}

\Question Uma empresa produz televisores de 2 tipos, tipo A (comum) e tipo B (luxo), e garante a restituição da quantia paga se qualquer televisor apresentar defeito 
grave no prazo de seis meses. O tempo para ocorrência de algum defeito grave nos televisores tem distribuição normal sendo que, no tipo A, com média de 10 meses 
e desvio padrão de 2 meses e no tipo B, com média de 11 meses e desvio padrão de 3 meses. Os televisores de tipo A e B são produzidos com lucro  de 1200 u.m. e 
2100 u.m. respectivamente e, caso haja restituição, com prejuízo de 2500 u.m. e 7000 u.m. Respectivamente.
\begin{tasks}
\task Calcule as probabilidades de haver restituição nos televisores do tipo A e do tipo B.
\task Calcule o lucro médio para os televisores do tipo A e para os televisores do tipo B.
\task Baseando-se nos lucros médios, a empresa deveria incentivar as vendas dos aparelhos do tipo A ou do tipo B?
\end{tasks}

\Question Um estudo comparou dois métodos (A e B) para ensinar matemática a alunos do primeiro grau. Após 10 semanas, o desempenho dos alunos foi 
avaliado em um teste. Teste a hipótese de que o método A resulta num melhor desempenho médio, ao nível $\alpha=5\%$, com base nos resultados da tabela a seguir:

\begin{table}[H]
\centering
\begin{tabular}{cccc}
\hline \hline
\multirow{2}{*}{Método}&Número   &Média      &Desvio padrão\\
                                       &de alunos&das notas&das notas        \\
\hline\hline
                                     A& 10           &8.15        &1.15                 \\
                                     B&   8           &7.31        &1.94                 \\
\hline \hline
\end{tabular}
\end{table}

\Question A lei trabalhista estabelece que o pagamento diário mínimo deve ser de $13,20$ U.M. (unidades monetárias). Assuma distribuição normal com desvio 
padrão igual a 2,0 U.M. Uma amostra aleatória de 40 trabalhadores de uma firma revelou média diária de 12,20 U.M .Esta firma deve ser acusada de estar 
infringindo a lei? Conclua a $1\%$ de probabilidade.

\Question A tabela a seguir mostra a frequência de acidentes automobilísticos por ano, de acordo som a faixa etária (idade) do motorista, para motoristas com 
idade inferior a $25$ anos. Teste a hipótese de que o número de acidentes independe da idade, a $5\%$ de probabilidade. Isto é, teste a hipótese de que o número 
anual de acidentes se distribui proporcionalmente nas faixas etárias.

\begin{table}[H]
\centering
\begin{tabular}{cccccc}
\hline \hline
\multicolumn{6}{c}{$\%$ de motoristas em cada faixa etária}\\
\hline\hline
$\%$ de motoristas& 10                &20              &20 & 25 & 25                 \\
idade (anos)            &15-16           &17-18        &19-20 & 21-22 & 23-24                 \\
\hline
número de acidentes &8 & 15 & 13 & 11 & 8 \\
\hline \hline
\end{tabular}
\end{table}

\Question Uma indústria farmacêutica conduziu um estudo para avaliar o tempo médio em dias para recuperação dos efeitos da gripe. O estudo comparou o tempo 
de indivíduos que tomaram $500$ mg diárias de vitamina C, contra  indivíduos que não tomaram vitamina C (nenhum suplemento). Com base nos dados a seguir, 
conclua e interprete a $5\%$ de probabilidade.

\begin{table}[H]
\centering
\begin{tabular}{ccc}
\hline \hline
                                      &\multicolumn{2}{c}{Tratamento}\\
																			\cline{2-3}
                                      &Nenhum suplemento&$500$mg Vit. C        \\
\hline\hline
Tamanho da amostra& 12           &12      \\
Tempo médio            & 7,4           &5,8     \\
Variâncias                   & 2,9          &2,4     \\
\hline \hline
\end{tabular}
\end{table}

\Question Um pesquisa de opinião entrevistou $50$ pessoas em dois distritos. O objetivo era verificar se a distribuição das opiniões era homogênea nos dois distritos. 
Com base nos dados da tabela, teste a hipótese de homogeneidade de opiniões usando $\alpha=5\%.$

\begin{table}[H]
\centering
\begin{tabular}{ccccc}
\hline \hline
                                      &\multicolumn{3}{c}{Opinião}&            \\
																			\cline{2-4}
                                      &Sim&Indeciso&Não             & Total  \\
\hline\hline
Distrito A           & 20           &9 & 21 & 50      \\
Distrito B           & 26           &3 & 21 & 50     \\
\hline
Total                  & 46           &12 &42&100    \\
\hline \hline
\end{tabular}
\end{table}

\Question Uma associação comercial afirma que o número médio de dias de trabalho perdidos anualmente, devido a problemas de saúde, é igual a 60. Uma extensa 
campanha educacional visando a conscientizar os trabalhadores quanto a importância de uma alimentação balanceada, higiene pessoal, prática de esportes 
etc, foi conduzida com o intuito de melhorar este quadro. Um ano após esta campanha, um estudo com $30$ trabalhadores forneceu média igual a $55$ dias. Assuma 
que o número de dias de trabalho perdidos anualmente é normalmente distribuído com variância $\sigma^{2}=275$. Pede-se:
\begin{tasks}
\task Pode-se afirmar que a campanha foi eficaz ao nível de $\alpha=1\%$ de probabilidade?
\task Para qual nível de significância se pode afirmar que a campanha educacional foi eficaz?
\end{tasks}

\Question Um gerente comercial acredita que um número excessivo de horas estejam sendo desperdiçadas em contatos comerciais, via telefone, entre os seus 
vendedores e os clientes em potencial. Ele deseja no máximo quinze horas por semana por vendedor. Este gerente comercial contratou uma empresa especializada 
para treinar seus vendedores. Após este treinamento, uma amostra de $36$ vendedores revelou média igual a $17h$ por semana por vendedor. O que pode ser concluído 
quanto a eficácia do treinamento? Assuma $\sigma^{2}=9$ e utilize $\alpha=5\%.$

\Question Com base em dados obtidos de $400$ mulheres, apresentados na tabela abaixo, pode-se concluir que o nível educacional e a adaptação à vida conjugal 
são independentes? Conclua a $5\%$ de probabilidade.

\begin{table}[H]
\centering
\begin{tabular}{ccccc}
\hline \hline
                                      &\multicolumn{4}{c}{Adaptação}\\
																			\cline{2-5}
Nível educacional&ruim&razoável&boa&muito boa\\
\hline\hline
         Universidade& 18  &29 & 70 & 115      \\
$2^{\circ}$ grau     & 17  &28 & 30 &    41     \\
$3^{\circ}$ grau     & 11  &10 & 11 &    20     \\
\hline
\end{tabular}
\end{table}

\Question Uma cooperativa de produtores possui uma máquina de encher vasilhame com um litro de leite. Para assegurar que em média cada vasilhame não terá 
leite a mais e nem a menos, o responsável pelo controle de qualidade amostra, semanalmente, $75$ vasilhames enchidos pela máquina. Se uma amostra fornecer 
$63,97$ litros e desvio padrão $s=0,25$ litros, deve-se parar a máquina para regulagem ou continuar a produção? Qual deve ser o procedimento adotado a $\alpha=5\%$ 
de probabilidade?

\Question A renda média de famílias com $4$ pessoas na região sudeste do Brasil, no ano de $1975$, era de $5$ U.M. Economistas acreditam que atualmente a renda 
média é maior. Pede-se,
\begin{tasks}
\task Quais seriam as hipóteses estatísticas $(H_{0} e H_{a}),$ para se tentar provar que atualmente a renda média é maior do que em 1975?
\task Quais são as informações necessárias para se realizar um \textbf{teste Z}?
\task Quais são as informações necessárias para se realizar um \textbf{teste t}?
\task Explique os dois possíveis erros (erro tipo I e erro tipo II) de decisão que podem ocorrer neste exemplo?
\end{tasks}

\Question Assuma que o consumo mensal per capita de determinado produto tem distribuição normal com desvio padrão igual a $5$ kg. Com a atual crise (do dólar, 
do apagão, do futebol...várias opções!) o departamento de vendas da fábrica decidiu que irá retirar o produto do mercado, caso o consumo médio $(\mu)$ per capita 
seja inferior a $10 kg.$ Se uma pesquisa de mercado, com uma amostra de $100$ indivíduos, revelar consumo médio mensal per capita de $9$ kg, pede-se: Qual deve 
ser a afirmação, ao nível de significância de $1,5\%$?

\Question No quadro abaixo estão as opiniões, com respeito ao desempenho e a potência do motor, de proprietários de veículos de um determinado fabricante. 
As opiniões foram classificadas pela idade do proprietário.

\begin{table}[H]
\centering
\begin{tabular}{ccc}
\hline \hline
                                      &\multicolumn{2}{c}{Opinião}\\
																			\cline{2-3}
Idade&Ruim&Bom\\
\hline\hline
Jovem       & 30  &20      \\
Experiente& 20  &30     \\
\hline
\end{tabular}
\end{table}

O que pode ser afirmado quanto à seguinte hipótese de nulidade? 

$H_{0}:$ Idade e opinião são independentes.

\Question Para comparar duas marcas de pará-choques, montaram-se seis de cada marca em $12$ carros compactos, fazendo-se cada carro colidir com um muro 
de concreto, a uma velocidade de $40$ km$\backslash$h. Registraram-se os seguintes custos de reparo:

\begin{table}[H]
\centering
\begin{tabular}{ccccccccc}
\hline \hline
Pára-choque&\multicolumn{6}{c}{Custo (R\$)}&Média & Variância\\
\hline
A       & 320  &310 & 380 & 360 & 320 & 345 & 339,17 & 744,17      \\
B       & 305  &290 & 340 & 315 & 280 & 305 & 305,80 & 434,17     \\
\hline \hline
\end{tabular}
\end{table}

Teste $(\alpha=5\%)$ a hipótese de igualdade entre os custos médios de reparo dos pará-choques.

\Question Se um dado não é viciado cada uma das seis faces ocorre com igual probabilidade. Um determinado dado foi lançado $720$ vezes, obtendo-se:

\begin{table}[H]
\centering
\begin{tabular}{c|ccccccc}
\hline \hline
Face                               &1     &2      & 3   & 4     & 5      & 6     &Total\\
\hline
Frequência observada & 129 &107 & 98 & 132 & 136 & 118 & 720 \\
\hline \hline
\end{tabular}
\end{table}
O dado será considerado viciado para qual nível de significância? Explique sua resposta. 

\Question O tempo médio, por operário, para executar uma tarefa, tem sido $100$ minutos. Introduziu-se uma modificação para diminuir esse tempo, e, após certo período, 
sorteou-se uma amostra de $16$ operários, medindo-se o tempo de execução de cada um. O tempo médio da amostra foi $85$ minutos, e o desvio padrão foi $12$ minutos. 
Estes resultados trazem evidências estatísticas da melhora desejada, considerando $\alpha=5\%$? Apresente as suposições teóricas usadas para resolver problema.

    \end{Exercise}

\end{document}